\documentclass[twoside,11pt]{article}

% Any additional packages needed should be included after jmlr2e.
% Note that jmlr2e.sty includes epsfig, amssymb, natbib and graphicx,
% and defines many common macros, such as 'proof' and 'example'.
%
% It also sets the bibliographystyle to plainnat; for more information on
% natbib citation styles, see the natbib documentation, a copy of which
% is archived at http://www.jmlr.org/format/natbib.pdf

\usepackage{jmlr2e}
%\usepackage{parskip}

% Definitions of handy macros can go here
\newcommand{\dataset}{{\cal D}}
\newcommand{\fracpartial}[2]{\frac{\partial #1}{\partial  #2}}
% Heading arguments are {volume}{year}{pages}{submitted}{published}{author-full-names}

% Short headings should be running head and authors last names
\ShortHeadings{95-845: AAMLP Proposal}{Khin and Leahy}
\firstpageno{1}

\begin{document}

\title{Heinz 95-845: Project Proposal}

\author{\name Kaung M. Khin \email kkhin/kkhin@andrew.cmu.edu \\
       \addr Heinz College\\
       Carnegie Mellon University\\
       Pittsburgh, PA, United States \\
       \AND
       \name Shawn Leahy \email sleahy/sleahy@andrew.cmu.edu \\
       \addr Heinz College\\
       Carnegie Mellon University\\
       Pittsburgh, PA, United States}
\maketitle
\section{Introduction}
Influenza, or simply the flu, is a seasonal contagious respiratory disease that
affects millions of people in the United States yearly with outcomes ranging
from just mild symptoms to even death. \citep{cdc} It is caused by the influenza virus with
the most infamous version in recent time being the 1918 flu pandemic nicknamed
the “Spanish Flu”. The economic cost of the flu on the United States is
estimated to be in the billions \citep{rolfes_etal._2017} so the Center for Disease Control (CDC) tracks
the seasonal flu using Flu Surveillance which is a collaborative effort between
the CDC and many healthcare providers. \\
The use of big data to track flu trends is not a novel idea \citep{ginsberg_2009}, in fact many
studies have attempted to learn from the mistakes of Google Flu Trends. The
challenges of using big data to attempt this type of prediction is
well-documented. \citep{lazer_kennedy_king_vespignani_2014} However, there has been some success in the use of social media
data and search data combined with health surveillance data to track and predict
the number of cases for the Zika virus outbreak in Latin America. \citep{mcgough_brownstein_hawkins_santillana_2017} We will be
using a variety of data sources to build a machine learning model to predict the
number of actual flu cases in each state.
\section{Analysis and Outcomes} \label{details}
\begin{itemize}
\item \textbf{Y:} Number of actual cases reported in a particular state/region
\item \textbf{U:} There is no applicable treatment for our study
\item \textbf{V:} We will use a number of data sources for our covariates but we will focus on the social media data, the health expenditure data and Google search data
\item \textbf{W:} People who have filled out the MEPs survey for a given year
\end{itemize} 
\section{Importance and Contribution to Existing Literature}
Demographic, hospital condition, and influenza surveillance data has been used
to predict occurrences and investigate trends for influenza infections on U.S.
military personnel. \citep{buczak_2016}  Machine learning methodologies have been used on medical
data and have been found to be successful in extrapolating from incomplete data \citep{chen_2017, santillana_2016}.
This paper seeks to merge these analyses together to predict influenza outbreaks
on a more diverse dataset utilizing machine learning methodologies that may have
to deal with incomplete information and using social media data to track changes
in infection rates.
\section{Methods}
\subsection{Data}
The Medical Expenditure Panel Survey is a set of large scale surveys of
individuals, families, providers, and employers across the United States.  MEPS
collects data health services, how frequently they are used, the cost of these
services, and how they are paid for, as well as data on the cost, scope, and
breadth of health insurance.  MEPS has been collected continually since 1996 \citep{meps}.
We will be utilizing the full year data files, medical conditions files, and
prescribed medicines files.  We will also be utilizing the Centers for Disease
Control FluView Portal data in order to validate our results.  This portal
accesses state, regional and national influenza statistics as well as mortality
information \citep{cdc_interactive}. We are also attempting to include social media data to better
track how fast and which geographic locations are experiencing influenza
infections.
\subsection{Modeling Approach}
Our approach involves utilizing demographic, location, medical condition, and
prescription drug information to predict susceptibility and frequency of
influenza outbreaks in regions of the United States.  We will create regional
models to classify if we predict an individual to contract influenza or not.
Variables for regional infection frequencies, infection rates, and hospital
visits due to influenza will be created.  Modeling approaches under
consideration include logistic regression, SVMs, decision trees (and random
forests), as well as neural networks.
\subsection{Evaluation Measures}
We will build a Naive Bayes model to predict influenza and this will be our
baseline model to compare against due to its assumption of conditional
independence.  We will measure the accuracy, sensitivity, specificity, and
F1-score for our models to try and gain insights into the results.  We need an
array of measures due to the fact that false positives and false negatives
should be given a different weight, but we also need to measure an overall model
accuracy of some sort.
\section{Limitations and Possible Avenues of Continuing Work}
This is not a real-time analysis tool that could be used ahead of time.
However, this could be used to iteratively improve preparedness year after year
as well as to track changes in disease spread over time.  Further analysis into
how disease spread changes year by year could be performed.  Also, our data
comes from a survey methodology and not EHRs or medical claims data.  This
introduces some uncertainty into every level of our analysis.
\bibliography{sample}
%\appendix
%\section*{Appendix A.}
%Some more details about those methods, so we can actually reproduce them.

\end{document}
